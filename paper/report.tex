\documentclass{article}

\usepackage[affil-it]{authblk}

\usepackage[USenglish,american]{babel}

\usepackage{amsfonts,amsmath,amsthm,amssymb}

\usepackage{tikz}
\usepackage{float}
\setlength{\parindent}{0mm}
\setlength{\parskip}{.5em}
\usepackage[letterpaper, margin=1in]{geometry}
%\usepackage{showframe}
\usepackage{graphicx}

\usepackage{titling}
\setlength{\droptitle}{-5em}   % This is your set screw


\usepackage{multicol,caption}
\setlength\columnsep{25pt}


\newenvironment{Figure}
  {\par\medskip\noindent\minipage{\linewidth}}
  {\endminipage\par\medskip}

\usepackage{enumerate}

\usepackage{verbatim}
\usepackage{listings}

\usepackage{siunitx}

\usepackage{color}

%%
%% Julia definition (c) 2014 Jubobs
%%
\lstdefinelanguage{Julia}%
  {morekeywords={abstract,break,case,catch,const,continue,do,else,elseif,%
      end,export,false,for,function,immutable,import,importall,if,in,%
      macro,module,otherwise,quote,return,switch,true,try,type,typealias,%
      using,while},%
   sensitive=true,%
   alsoother={$},%
   morecomment=[l]\#,%
   morecomment=[n]{\#=}{=\#},%
   morestring=[s]{"}{"},%
   morestring=[m]{'}{'},%
}[keywords,comments,strings]%

\lstset{%
    language         = Python,
    basicstyle       = \tiny\ttfamily,
    keywordstyle     = \bfseries\color{blue},
    stringstyle      = \color{magenta},
    commentstyle     = \color{red},
    showstringspaces = false,
    backgroundcolor  = \color{lightgray},
    numbers          = left,
    title            = \lstname,
    numberstyle      = \tiny\color{lightgray}\ttfamily,
    inputencoding=utf8,
    extendedchars=true,
    literate={á}{{\'a}}1 {ã}{{\~a}}1 {é}{{\'e}}1,
}

\begin{comment}
\lstset{
    language=bash, %% Troque para PHP, C, Java, etc... bash é o padrão
    basicstyle=\ttfamily\small,
    numberstyle=\footnotesize,
    numbers=left,
    backgroundcolor=\color{gray!10},
    frame=single,
    tabsize=2,
    rulecolor=\color{black!30},
    title=\lstname,
    escapeinside={\%*}{*)},
    breaklines=true,
    breakatwhitespace=true,
    framextopmargin=2pt,
    framexbottommargin=2pt,
}
\end{comment}

\usepackage{xspace}
\usepackage{url}
\usepackage{cite}

\usepackage{coffee4}

\newcommand{\Bold}{\mathbf}


\title{Term Research \\ Unlabeled Acoustic Anomaly Detection}

\date{\today}
\author{Philip Robinson}
\affil{Oregon Health Sciences University}

\begin{document}


\maketitle

\def\aa{acoustic anomalies\xspace}
\def\ae{Auto-Encoders\xspace}
\def\vae{Variational \ae\xspace}
\def\gumiho{Gumiho Network\xspace}
\def\condgen{Conditional Generating Network\xspace}
\def\desc{Descriminator Network\xspace}

\def\np{Neyman-Pearson Lemma\xspace}
\def\pdf{probability density function\xspace}
\def\fpr{false positive rate\xspace}
\def\tfpr{true/false positive rate\xspace}
\def\tpr{true positive rate\xspace}
\def\keras{\texttt{keras}\xspace}
\def\pytorch{\texttt{pytorch}\cite{paszke2017automatic}\xspace}
\def\tensorflow{\texttt{tensorflow}\xspace}
\def\ray{\texttt{ray}\xspace}

\def\encoder{\texttt{encoder}\xspace}
\def\decoder{\texttt{decoder}\xspace}
\def\bottle{\texttt{bottle}\xspace}
\def\GMM{\texttt{GMM}\xspace}



\begin{abstract}
University of Hawaii's Aloha Cabled Observatory (ACO) has approximately 10 years
of unlabeled hydrophone data, in need of indexing, labeling, and cleaning; to more
easily extract \aa.
Currently searching the audio recordings is done at human pace by listening
to the audio stream, or looking at it's spectrogram, which takes $\sim$ 2-20 minutes
for every 5 minutes of audio. This is intractable at scale.
Unfortunately, audio data is known to be high dimensional, making it
very difficult to label raw input streams. Additionally, the properties of
\aa are not necessarily known a priori. It is known that unsupervised anomaly detectors,
leveraging \vae are successful for acoustic anomaly detection\cite{Koizumi:2019:UDA:3282584.3301702}.
This is an exploration of one of these models as a means to identify when whales
are vocalizing.

\end{abstract}

\begin{multicols}{2}
\section{Introduction}

Scientists, with University of Hawaii's Aloha Cabled Observatory (ACO),
have gathered 10+ years of continuous audio for acoustic ocean survey studies.
Currently searching the audio recordings is done at human pace by listening
to the audio stream, which takes $\sim$ 2-20 minutes for every 5 minutes of
audio, depending on experience levels. Even at its fastest, this is intractable
at scale. Often ocean hydrophone observatories are used to survey and track
cetations vocalizations for population measurement, acoustic analysis for
earthquake/tectonic events, and applications in monitoring vessel traffic
and activities.
The ACO hydrophone data is unlabeled and growing at 1.2 terabytes per year.

\begin{quote}
{\small{\em ``Being able to automatically detect whale calls in an un-monitored system can help
identify time-series of whale locality to the area. Being able to apply this
detection algorithm to a long time-series of sub-sea audio, such as that from
the ACO, allows scientists to derive a time-series of whale activity in the
area. This can help us study the relationship between migration patterns and
known climate events.''}

- Kellen Rosburg\\Senior Ocean Computer Specialist\\OOI Cabled Array, APL/UW}
\end{quote}

In the most general form, it would be extremely useful to have a generic unsupervised
acoustic anomaly detector, for hydrophone data. This would allow a human or machine
reviewer to focus on only interesting data in the task of event sorting and classification.

\section{Model}

In prior work on acoustic anomaily detection\cite{1810.09133}
%\texttt{
%  Unsupervised Detection of Anomalous Sound based on Deep Learning
%  and the Neyman-Pearson Lemma}
address the generic task of
anomaly detection, over image encoded \aa.
A \vae is expected to behave as an identity function, however reconstruction error can
be an indicator of content unseen in the training data.
When a \vae is trained on typical data, a threshold over an anomaly score can be used
for to flag atypical data. The paper proposes a minimization of \fpr, given a constrained
\tpr, by expressing a threshold of reconstruction error.

Proposed is an anomaly generating network. This is done by fitting
the \vae{}'s gaussian manifold on general data, generating an under-specified \pdf (by
minimizing KL divergence), then tighter fitting a more descriptive \pdf
over the latent vectors of typical data projected into the gaussian manifold. Once both
distributions are fit, rejection sampling in the form of sampling from the general
distribution values of low probability from the descriptive distribution.

Training is broken up into thee major phases. These phases contribute to
fit the generator and discriminator networks, as well as a typical data model
(in this case our GMM) for informing rejection sampling.

The first phase updates the parameters of the encoder and the generator for all
input data. The goal of this process is to define the general \pdf of the latent
space of all data and a mapping to a unit gaussian manifold.

%In the case of ACO
%data, this can be accomplished by sampling high traffic periods of time
%(like March for Blue Whales).

The second phase updates the parameters of the \encoder and \texttt{descriminor} with a
loss function that is defined by reconstruction error against a threshold,
and an approximation of \tfpr{}s (which cannot be calculated directly). This phase
requires sampling \aa from the unit gaussian and rejecting likely events measured by
the specific \pdf, through the generator network.
For the ACO typical data can be sampled from the evenings or low traffic months, as
cetations tend to not be as vocal in the evenings; this strategy is also resilient to
accumulative damage/drift of the equipment over time.

The third phase is to update the specific \pdf modeled by a \GMM. This is repeated as necessary.

Discrimination is accomplished by learning an appropriate threshold for the likelihood
of a latent representation of an event.
\end{multicols}

\includegraphics[width=\textwidth]{./model.png}

\begin{multicols}{2}

\section{Topics}

Due to the complex nature of this architecture, it is important to provide an
understanding and background of supporting topics. This section introduces the
pre-requisite knowledge to understand the final model.

\subsection{\ae}
\ae are an unsupervised neural network model that attempts to constrain an
identity function. This is usually done by optimizing the weights of the neural
network to reconstruction error, while passing through a low dimensional \bottle.

\[J^{R}=||Input - Output||^2_2\]

The \bottle, represented as a narrowing of nodes in the neural network, expresses
a compressed representation of the original content wrt it's neighboring encoding and
decoding networks. These networks can be thought of as learning a lossy compression
and decompression functions. The lossy characteristic of these mapping functions is
expressed in image data as a blurry recreation of the original content.

\resizebox{.8\columnwidth}{!}{\input{./ae.tex}}

\lstinputlisting[firstline=146, lastline=185, firstnumber=146]{../ae.py}


\subsection{\vae}

\vae are similar to \ae, however they restrict the form of manifold forming the
\bottle to a gaussian manifold. It is important to note that trained \vae
can be split at the \bottle, and used like a generative model. This is
done by sampling from a unit gaussian; the samples is interpreted as a latent
representation of an input from the \encoder. The \decoder is then used on this sample
to generate new content that would have been encoded as the latent representation.

In order to train this style of network, the loss function is complimented by
Kullback Leibner divergence, to constrain the latent space to a gaussian manifold.

\[J^{KLR}=J^R + KL(z||\mathcal{N}(0, 1))\]

\resizebox{.8\columnwidth}{!}{\input{./vae.tex}}
\lstinputlisting[firstline=23, lastline=90, firstnumber=23]{../vae.py}

\subsection{Rejection Sampling}

Rejection sampling is a technique used in many statistical models. Many distributions
are difficult to sample from, however simple to evaluate likelihood against. A second
distribution whose domain is unbiased wrt the original distribution may be elected to
act as a proxy for sampling.

The simplest example of this is in attempting to sample points from a unit circle. There
exist known algorithms to sample from a uniform distribution. A two dimensional uniform
distribution is equivalent to sampling from a rectangle. Rejection sampling can be used
to sample from the unit circle by electing a 2 dimensional uniform distribution that
encompasses the area of the target circle, and rejects points under the constraint that
$x^2 + y^2 \le 1$.

This relates to \vae because there exists known algorithms for sampling from gaussian
distributions. We can elect any statistical model to describe typical data, that
shares an unbiased domain with a gaussian. This typical distribution can then be used
to form a likelihood threshold for rejection criteria. This strategy informs the core
of the proposed model, further explained by the \np.

\subsection{\np}

The \np states that the likelihood ratio is the `uniformly most powerful discriminator'
for a statistical hypothesis test. This is leveraged into forming a loss function by
increasing the reconstruction error of anomalous data, and decreasing the reconstruction
error of typical events in data.

\[J^{NP} = FP - TP\]

Theoretically a discriminator network leveraging this loss function, with known anomalous
and typical data, will become better identity function for typical data and increasingly
poor at reconstructing atypical data.

\section{\gumiho{s}}

Although this term isn't in the common nomenclature, a gumiho network refers to a network
that has more than one tail, generating output. The intent is to use different loss functions depending
on which network output is triggered. For \ae and \vae this allows for the learned
encoding function to generate the same manifold, while satisfying very different loss
enforced constraints. It is helpful, although not necessary, to choose non-competing
loss functions.

\resizebox{\columnwidth}{!}{\def\layersep{2.5cm}

\begin{tikzpicture}[shorten >=1pt,->,draw=black!50, node distance=\layersep]
    \tikzstyle{every pin edge}=[<-,shorten <=1pt]
    \tikzstyle{neuron}=[circle,fill=black!25,minimum size=17pt,inner sep=0pt]
    \tikzstyle{input neuron}=[neuron, fill=green!50];
    \tikzstyle{output neuron}=[neuron, fill=red!50];
    \tikzstyle{discriminator neuron}=[neuron, fill=purple!50];
    \tikzstyle{threshold neuron}=[neuron, fill=pink!50];
    \tikzstyle{hidden mu}=[neuron, fill=orange!50];
    \tikzstyle{hidden sigma}=[neuron, fill=black!50];
    \tikzstyle{hidden neuron}=[neuron, fill=blue!50];
    \tikzstyle{annot} = [text width=4em, text centered]

    % Draw the input layer nodes
    \foreach \name / \y in {1,...,7}
        \node[input neuron, pin=left:Input \#\y] (I-\name) at (0,-\y) {};

    % Draw the hidden layer nodes
    \foreach \name / \y in {1,...,3}
        \path[yshift=-4cm]
            node[hidden mu] (M-\name) at (\layersep, \y cm) {};

    % Draw the hidden layer nodes
    \foreach \name / \y in {1,...,3}
        \path[yshift=-8cm]
            node[hidden sigma] (S-\name) at (\layersep, \y cm) {};

    % Draw the hidden layer nodes
    \foreach \name / \y in {1,...,3}
        \path[yshift=-6cm]
            node[hidden neuron] (H-\name) at (\layersep+\layersep, \y cm) {};

    \foreach \name / \y in {1,...,7}
        \path[yshift=0cm]
            node[output neuron, pin={[pin edge={->}]right:Output}] (O-\name) at (\layersep+\layersep + \layersep,4-\y) {};

    \foreach \name / \y in {1,...,7}
        \path[yshift=0cm]
            node[discriminator neuron] (D-\name) at (\layersep+\layersep+\layersep,-4-\y) {};

    \node[threshold neuron, pin={[pin edge={->}]right:Discriminate}] (T) at (\layersep+\layersep+\layersep+\layersep,-7) {};


    % Connect every node in the input layer with every node in the
    % hidden layer.
    \foreach \source in {1,...,7}
        \foreach \dest in {1,...,3}
            \path (I-\source) edge (M-\dest);

    \foreach \source in {1,...,7}
        \foreach \dest in {1,...,3}
            \path (I-\source) edge (S-\dest);

    \foreach \source in {1,...,3}
        \path (S-\source) edge (H-\source);

    \foreach \source in {1,...,3}
        \path (M-\source) edge (H-\source);

    % Connect every node in the hidden layer with the output layer
    \foreach \source in {1,...,3}
        \foreach \dest in {1,...,7}
            \path (H-\source) edge (O-\dest);

    % Connect every node in the hidden layer with the output layer
    \foreach \source in {1,...,3}
        \foreach \dest in {1,...,7}
            \path (H-\source) edge (D-\dest);

    \foreach \source in {1,...,7}
        \path(D-\source) edge (T);


    % Annotate the layers
    \node[annot,above of=H-1, node distance=5cm] (hl) {${}_{z\sim\mathcal{N}(0, 1)}$};
    \node[annot,left of=hl] {$\mu / \sigma$};
\end{tikzpicture}
}

Since the discriminator network uses a threshold over reconstruction error, the \vae
(that uses only recognition error) is not considered a non-competing loss function.
This conditional training, dependent on the evaluation path lends itself to \pytorch
over \keras. This will be discussed in greater detail in the next section.


\section{Implementation}

Systems like \keras and \tensorflow use a precompiled execution graph, to facilitate
more predictable back propogation. For this model, there is a parameterized
discriminating function in the \bottle of the network. Additional complexity is added,
due to the sharing \encoder variables across multiple \decoder networks. \pytorch doesn't
restrict the model to precompiled execution graphs and allows for use of the
\texttt{torch.no\_grad()} context manager in order to restrict when variables can be
treated to not need back propogation.

The user provides an \encoder, \decoder, the interfacing dimension size $h$,
the \bottle dimmension size $z$, and the count of mixtures for the \GMM $M$. The \encoder
and \decoder are expected to reduce the input to and from a linear layer of size $h$.
A linear \bottle layer is provided to map from $h$ to $z$
dimmensions and extract $\mu$ and $\sigma$ to inform KL loss by reparameterization. The
\decoder is then preceded by a provided layer $\eta$ that maps back from $z$ to $h$
dimmensions. This abstraction layer makes it much easier to modularize the network.

The code represents a direct inheritance principled on object oriented design of machine
learning models. \desc is a \condgen, is a \gumiho, is a \vae, is a \ae.

\subsection{\gumiho}

The \gumiho is the first unfamilure network abstraction, and therefore starts the descriptive
sections of this text.

The \gumiho is a \vae with multiple goal states. A user can add additional \decoder{s}
and cooresponding loss functions. This allows a trained model to develop an embedding
optimized, not only for reconstruction but, for various additional goals like classification.
The \texttt{add\_tail(self, network, loss)} method requires the network can be fed by an $h$
dimensional linear layer.

\lstinputlisting[firstline=14, lastline=56, firstnumber=14]{../gumiho.py}


\subsection{\condgen}

In a \vae, the embedding is mapped to a gaussian manifold. This means that sampling from
a unit gaussian and passing data through a \decoder will generate results cooresponding to the
\decoder's goal. When used with the tail associated with the reconstruction task, this produces
viable input data. the \condgen allows condition to be provided to the sampled value and the
tail elected to be specified by the user. For the final model this is a mechanism to allow the
\GMM to be used for conditioning and the reconstructor to be used to generate anaomolies
as found in the reference materials.

\lstinputlisting[firstline=37, lastline=65, firstnumber=37]{../gumiho_descriminator.py}

\subsection{\desc}

The \desc models the machine specified\cite{1810.09133} for unsupervised anomaly detection.
There are three tails, one for training the \GMM, one for discrimination, and one for
reconstruction.

\lstinputlisting[firstline=37, lastline=68, firstnumber=37]{../gmm.py}

The code for this model is found in the repository instead of this paper due
to its complexity.


\section{Data}

The model is build to be agnostic to types of \encoder and \decoder. The expectation of the
user is to provide models appropriate for their data. It is also expected that a loading
\texttt{coroutine} is provided. The coroutine needs to take in the count of a batch in order
to be trained in a consistant way.

\subsection{MNIST}
\def\mnist{\texttt{MNIST}\xspace}

The \mnist data hopes to example hand written single digit entries. The
original intent is to inform algorithms of had written digit recognition for the postal
service. For this term project, \mnist was used to more consistently and reproducibly
explore model building.

\lstinputlisting[firstline=188, lastline=231, firstnumber=188]{../ae.py}

In this trained model, the digits listed as $\{1, 7, 4\}$ were labeled as anomalous. Any other
digit were described as typical events. The expectation of a trained network is to
generate high accuracy reconstructions of $\{2,3,5,6,8,9\}$ and low accuracy reconstructions of
$\{1, 7, 4\}$ compile.

\begin{multicols}{3}
\includegraphics[width=\columnwidth]{./vae-images/C.png}
\includegraphics[width=\columnwidth]{./vae-images/E.png}
\includegraphics[width=\columnwidth]{./vae-images/I.png}
\end{multicols}

The images above are sources and their paired results from passing the \mnist data through
the trained \vae with loss of mean squared error. It is apparent by adjusting the networks
that using convolutions and ending the networks with sigmoid result in much higher quality
yields.




\subsection{Aloha Cabled Observatory}

The ACO data is gathered at N$22^\circ45.110'$ W$158^\circ00'$. The recordings are
encoded as, custom, variable bit-width raw peak sensor readings. Each file is saved
as a 5 minute duration (barring hardware related problems), named with it's datetime
stamp. The set elected for this project was 44100 samples per second. The developed
\texttt{ACOio.py} library allows simple manual indexing and datetime-search of the data.



%\begin{Figure}
\begin{lstlisting}
from aco import ACOio, datetime, timedelta

loader = ACOio('./basedir/')
target = datetime(
    day=18, month=2, year=2016,
    hour=7, minute=55
)

src = loader.load(datetime)
snip = src[
  timedelta(seconds=7):
  timedelta(seconds=11)
]
snip.View()
\end{lstlisting}
%\captionof{figure}{ACOio}
%\label{fig:acoio}
%\end{Figure}



\begin{Figure}


  \includegraphics[width=\columnwidth]{./rawwave.png}
  \captionof{figure}{Raw Data}
  \label{fig:rawwav}
\end{Figure}

\newcommand{\figsite}[1]{\texttt{(Figure~\ref{#1})}\xspace}

It is visible, from \figsite{fig:rawwav}
that the direct current gain is not centered at zero, nor trivially accumulative. This
is a consequence of changes in atmospheric pressure, due to the ocean's motion,
effecting the signal. %A high pass filter is used to remove this, whose parameters were
%empirically elected.

\begin{comment}
  \begin{Figure}
  \includegraphics[width=\columnwidth]{./rawvocal.png}
  \captionof{figure}{Raw Vocalization}
  \label{fig:rawvoc}

  \end{Figure}
 \end{comment}

It is also not obvious this track has a vocalization, highlighted in \figsite{fig:rawvoc}.
This pattern is indicative of high amounts of noise, and is expected for all samples.

\begin{Figure}
  \includegraphics[width=\columnwidth]{./rawspec.png}
  \captionof{figure}{Raw Spectrogram}
  \label{fig:rawspec}
\end{Figure}


\begin{comment}

\subsection{Spectral Subtraction}

Spectral subtraction is a technique used to denoise signals with additive white noise.
It is also, a very cheap algorithm with many variants.~\cite{specsub2008}.

The sampled signal $y$ is modeled as the desired signal $x$ and background noise $b$.
\[y[n] = x[n] + b[n]\]
At a single frame at point $p$ and length $L$,

\begin{align*}
  Y(pL, \omega) &= X(pL, \omega) + B(pL, \omega)\\
  |Y(pL, \omega)|^2 &= |X(pL, \omega)|^2 \\&+|B(pL, \omega)|^2 + |X(pL, \omega)*B(pL, \omega)|\\
\end{align*}

Background noise can be estimated, given sufficiently small frames, by subtracting
the mean power spectrum from a segment of only noise. Using Welch's method~\cite{welch1976}
we are able to estimate the mean power spectrum. This use implements an algorithm
allowing for adjustable $\alpha$ and $\beta$ parameters, used in tuning some audio artifacts
consequent of spectral subtraction ~\cite{Berouti1979EnhancementOS}.
In later work, these could be learned parameters.
\end{comment}

Inspired by these plots, and studies in signal processing, the audio track can be represented
as an image.  Expressing acoustic signals as spectrogram and mel-frequency cepstrum algorithms,
is a common way to enable these models. This representation lends itself to many deep learning
models. Additionally, this allows reasonable evaluation of a target model by using any well
studied image dataset. For this project the target dataset is described in the next subsection.


\section{Results}

With a small enough dataset and low enough embedding dimmension, the model trains. It is observable
that training decresses loss making \tpr and \fpr seperate, respecting the expected behavour by the
\np. It is also clear that changing out the \encoder and \decoder networks between linear and
convolutional significanly improves performance for reconstructions. Finally, tailing the \decoder
network with a \texttt{sigmoid} layer improves performance as well. As an alternative to using
\texttt{sigmoid} finishing layer, the model can be traned on \texttt{binary cross entropy} loss,
but in conjunction seems to have no better performance.

The \GMM has underflow errors when appropriate dimmensions for the ACO dataset are selected.
To treat this the forward pass generates log-likelihoods, however there are complications in
the update step that prevent continuing to work in log space, once we add the $\phi_i$
parameters.


\lstinputlisting[firstline=71, lastline=165, firstnumber=71]{../gmm.py}

The code can be found in it's current state on github.\footnote{\texttt{https://github.com/probinso/gumiho-network}}
A webUI was attempted to manage the dataset, and allow for exploration. This web interface
has led me to believe that the \texttt{ACOio.py} library is far more effective mixed with
\texttt{jupyter} than with a standalone server.\footnote{\texttt{https://github.com/probinso/BlueScienceFactory}}

\bibliography{references}{}
\bibliographystyle{plain}

\end{multicols}

\end{document}
